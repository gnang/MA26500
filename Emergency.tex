\documentclass[10pt]{article}
\begin{document}
\noindent\parbox{7cm}{\large Purdue University\\
                    Fall 2014}
\hfill
\parbox{7cm}{\large Instructor: Edinah Gnang\\
                    Office: 646 MATH\\
                    E-mail: {\tt egnang@purdue.edu}}

 
\begin{center}                   EMERGENCY PREPAREDNESS BRIEFING
\end{center}
Purdue University is a very safe campus and there
is a low probability that a serious incident will occur here at Purdue. 
However, just as we receive a “safety briefing” each time we get on an
aircraft, we want to emphasize our emergency procedures for evacuation and
shelter in place incidents. Our preparedness will be critical if an
unexpected event occurs.

Emergency preparedness is your personal responsibility. Purdue University
is continuously preparing for natural disasters or human-caused incidents
with the ultimate goal of maintaining a safe and secure campus. Let’s
review the following procedures:

\begin{itemize}
\item  To report an emergency, call 911.

\item  To obtain updates regarding an ongoing emergency, and to sign up for
    Purdue Alert text messages, view {\tt www.purdue.edu/ea}

\item There are nearly 300 Emergency Telephones outdoors across campus
and in parking garages that connect directly to the Purdue Police Department
(PUPD).  If you feel threatened or need help, push the button and you will
be connected immediately.

\item EMERGENCY NOTIFICATION PROCEDURES are based on a simple concept:
  hear a fire alarm inside, proceed outside. If you hear a siren outside,
proceed inside.

\item EMERGENCY RESPONSE PROCEDURES:

\begin{itemize}
\item Review the Emergency Procedures Guidelines
\begin{verbatim}
https://www.purdue.edu/emergency_preparedness/flipchart/index.html
\end{verbatim}

\item Review the Building Emergency Plan (available from the building deputy)
for:
\begin{itemize}  
\item evacuation routes, exit points, and emergency assembly area

\item when and how to evacuate the building.

\item  shelter in place procedures and locations

\item additional building specific procedures and requirements.
\end{itemize}
\end{itemize}

\item EMERGENCY PREPAREDNESS AWARENESS VIDEOS
``Shots Fired on Campus: When Lightning Strikes," is a 20-minute active
shooter awareness video that illustrates what to look for and how to prepare
and react to this type of incident. See 

\begin{verbatim}
  http://www.purdue.edu/securePurdue/news/2010/
       emergency-preparedness-shots-fired-on-campus-video.cfm
\end{verbatim}

(Link is also located on the EP website)

\item If we hear a fire alarm, we will immediately suspend class,
evacuate the building, and proceed outdoors, and away from the
building. Do not use the elevator. Indoor Fire Alarms mean to stop
class or research and immediately evacuate the building. Proceed to
your Emergency Assembly Area away from building doors (Outside 
between UNIV and BRNG). Remain outside until police, fire, or other 
emergency response personnel provide additional guidance or tell you 
it is safe to leave.

\item All Hazards Outdoor Emergency Warning Sirens mean to immediately
seek shelter (Shelter in Place) in a safe location within the closest
building. ``Shelter in place'' means seeking immediate shelter
inside a building or University residence. This course of action may
need to be taken during a tornado, a civil disturbance including a
shooting or release of hazardous materials in the outside air. Once
safely inside, find out more details about the emergency. Remain in
place until police, fire, or other emergency response personnel
provide additional guidance or tell you it is safe to leave.

\item  If we are notified of a Shelter in Place requirement for a tornado
    warning, we will suspend class and shelter in the lowest level of this
    building away from windows and doors.

\item  If we are notified of a Shelter in Place requirement for a hazardous
    materials release, or a civil disturbance, including a shooting or other
    use of weapons, we will suspend class and shelter in our classroom,
    shutting any open doors or windows, locking or securing the door, and
    turning off the lights.

Additional preparedness information as it might impact this class is
available on the syllabus and the Emergency Preparedness website
\begin{verbatim}http://www.purdue.edu/ehps/emergency_preparedness/index.html
\end{verbatim}
\end{itemize}
\end{document}

