\documentclass[12pt]{article}
\pagestyle{empty}
\setlength{\textwidth}{7in}
\setlength{\oddsidemargin}{-0.5in}
\setlength{\topmargin}{-1.0in}
\setlength{\textheight}{9.5in}

%\setlength{\evensidemargin}{0in}
%\setlength{\oddsidemargin}{0in}
%\setlength{\textwidth}{6.5in}

\begin{document}

\begin{center}
{\large\bf Teaching Assistant's Information for MA 265}
\end{center}

You should maintain the grades using Blackboard or Webassign.  

The homework assignments are posted at the WebAssign web site:\\
{\tt http://www.webassign.net/purdue/login.html}.

I will be collecting homework each Friday in class

I will put the homework in your mailbox each Friday by 5:45 (please give me the
number and combination for your mailbox).
Please bring the graded homework to my office (put it under the door or lean
it against the door).
I would like you to return it to me by 3:00 on Monday; once or twice during the
semester it's OK to return it on Wednesday by 3:00.

Note that the answers to {\it all} problems are at the end of the book.  Thus a
student should get 0 if they only give the answer without showing how they got
it.

I will include a note telling you which problems you should grade.  Grade 
each of these problems on a scale of 0--5, and mark the score (for example, 3/5
or 4/5) next to the problem.  
Put the total score at the top of the first page (for example, if the student 
got 23 points out of 25 possible points put ``23/25'' at the top of the first 
page).  {\bf Please follow these directions precisely, to avoid confusing the 
students.}

For problems with several parts (a,b,c,\dots) the parts should be graded as
separate problems (so each part that is graded counts up to 5 points.

You should {\it not} accept late homework---put a note on the first page saying
``Late---Not Graded'' and return it to me with the rest of the papers.

Occasionally I will ask you to excuse a student from a homework assignment.
Please keep track of this. At 
the end
of the semester, the student's final homework grade will be their total 
number of homework points, divided by the total possible homework points in the
assignments they were {\it not} excused from, and then multiplied by 100 (since
homework counts 100 in the final grade).

I have asked them to do the problems in order and to staple the pages together.
If a student does not do this, you should put a warning on their paper the
first time, and after that you should deduct 3 points on each 
assignment when this happens again.

If you have questions about how to grade any problem, please ask me about it, 
If you can't find me in my office (I'm in most of the time), you're welcome to
call me at home (497-4532).

{\bf Sometimes you should give a 0 for a problem.}  In general you should only
give points for progress towards a correct solution.  This means that if they
start out completely wrong they may get a zero.  Similarly, if they make a
fatal mistake part way through a problem, you ordinarily shouldn't give any
points for the rest of their work on that problem.  If you're unsure about a
particular paper, please ask me about it.

\end{document}
